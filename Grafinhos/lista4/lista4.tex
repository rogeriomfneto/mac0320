\documentclass[11pt,a4paper,notitlepage]{exam}
\usepackage[utf8]{inputenc}
\usepackage{graphicx, wrapfig}

\usepackage{amsmath}
\usepackage{amsthm}
\usepackage{amssymb}
\usepackage{mathtools}
\usepackage[shortlabels]{enumitem}

\renewcommand*{\proofname}{Prova}
% bold math
\usepackage{amsbsy}

% draw pictures (and graphs)
\usepackage{tikz}

% \usepackage[usenames,dvipsnames,svgnames,table]{xcolor}

% code in latex
\definecolor{dkgreen}{rgb}{0,0.6,0}
\definecolor{gray}{rgb}{0.5,0.5,0.5}
\definecolor{mauve}{rgb}{0.58,0,0.82}
\definecolor{newink}{rgb}{0,0.1,0.25}
\usepackage{caption}
\usepackage{listings}
\lstset{frame=tb,
  language=Python,
  aboveskip=3mm,
  belowskip=3mm,
  showstringspaces=false,
  columns=flexible,
  basicstyle={\small\ttfamily},
  numbers=none,
  numberstyle=\tiny\color{gray},
  keywordstyle=\color{blue},
  commentstyle=\color{dkgreen},
  stringstyle=\color{mauve},
  breaklines=true,
  breakatwhitespace=true,
  tabsize=3
}


\usepackage{multirow}

% definition equal
\newcommand\eqdef{\mathrel{\overset{\makebox[0pt]{\mbox{\normalfont\tiny\sffamily def}}}{=}}}

% independence equal
\newcommand\eqindep{\mathrel{\overset{\makebox[0pt]{\mbox{\normalfont\tiny\sffamily indep}}}{=}}}


% independent and identically distributed equal
\newcommand\eqiid{\mathrel{\overset{\makebox[0pt]{\mbox{\normalfont\tiny\sffamily i.i.d.}}}{=}}}

% * to cdot
% \mathcode`\*="8000
% {\catcode`\*\active\gdef*{\cdot}}

% pseudo-code
\usepackage[portuguese, linesnumbered]{algorithm2e}
\newcommand\Recebe{\leftarrow}
\newcommand\Comment{\vartriangleright}
\SetKw{Devolva}{devolva}
% Example:
% \paragraph{}
% \SetAlgoNoLine
% \textsc{Título-Do-Algoritmo}($A, n$)\\
% \begin{algorithm}[H]
%   \Devolva $A$
% \end{algorithm}
%

% pair ceil
\DeclarePairedDelimiter{\ceil}{\lceil}{\rceil}

% pair ceil
\DeclarePairedDelimiter{\floor}{\lfloor}{\rfloor}

% images
\usepackage{graphicx}
\graphicspath{ {./} }
% use: \includegraphics[scale=1]{image}


\setlength{\parindent}{3em}
\setlength{\parskip}{0.5em}

\begin{document}
% \SetAlgoNoLine
\begin{center}
  %NOME E NUSP
  Nome: Rogério Marcos Fernandes Neto\hphantom{xxx} NUSP: 10284632\\
  %CURSO
  Curso: Bacharelado em Ciência da Computação\\
  %MATÉRIA
  MAC0320 - Introdução à Teoria dos Grafos
  \paragraph{}
  \textbf{LISTA 4}
\end{center}
\paragraph{E16.} Provar (nos moldes da prova vista em aula para o algoritmo de Kruskal) que o algoritmo
descrito a seguir constrói uma árvore geradora de custo mínimo.\\

\noindent
\underline{ALGORITMO DESAPEGADO}\\

\noindent
Entrada: Grafo conexo $G = (V, A)$, com custos $c_a$ em cada aresta $a \in A$.\\
Saída: Árvore ótima $T$ (árvore geradora de custo mínimo).\\

\noindent
1. (Ordenação) Ordene as arestas de G em ordem não-crescente de seus custos. Chame-as de \phantom{xxx}$a_1 , a_2 , \dots , a_m$ , sendo $c(a_1 ) \geq c(a_2 ) \geq \dots \geq c(a_m )$.\\
2. $T \Recebe G$.\\
3. Para $i = 1$ até $m$ faça\\
\hphantom{xxx}se $T - a_i$ é conexo então $T \Recebe T - a_i$\\
4. Devolva $T$

\paragraph{Solução:}
\begin{proof}
  Seja $G$ um grafo e seja $T$ a árvore geradora de custo mínimo constuída pelo algoritmo DESAPEGADO.\\
  \hphantom{xxx}Notemos, primeiramente, que $T$ realmente é uma árvore geradora. O algoritmo remove arestas enquanto o grafo resultante for conexo. Portanto, $T$ é um grafo conexo minimal, ou seja, é uma árvore.\\
  \hphantom{xxx}Iremos provar agora que $T$ realmente é ótima. Seja $A(T)=\{e_1, \dots, e_k\}$, onde $c(e_i) \geq c(e_j)$ se $i < j$. Seja $T^*$ uma árvore geradora ótima de $G$ com mais arestas em comum com $T$. Suponha, por absurdo, que $T \neq T^*$. \\
  \hphantom{xxx}Seja $e_j = uv$ a primeira aresta em $A(T)$ tal que $e_j \notin A(T^*)$ (isso é, $\{ e_1, \dots, e_{j-1}\} \in A(T^*)$) e seja $P$ o único caminho em $T^*$ que conecta $u$ a $v$. Como $e_j$ não foi removida da árvore $T$ significa que no momento de avaliação de $e_j$, $e_j$ era uma ponte e portanto, pelo menos alguma aresta $xy$ diferente de $e_j$ no circuito formado por $P+e_j$ foi removida antes do algoritmo chegar a $e_j$. Como $xy$ está em $P$ então $xy$ está em $T^*$ e como $xy$ foi avaliada antes de $e_j$ então temos que $c(xy) \geq c(e_j)$.\\
  \hphantom{xxx}Seja $T' = T^* - xy + e_j$. Note que $T$ é uma árvore, pois foi mantida a quantidade de arestas e conexidade. Além disso, note que $c(T') = c(T^*) - c(xy) + c(e_j) \leq c(T^*)$, mas como $T^*$ é ótima então temos que $c(T') = c(T^*)$. Portanto $T'$ também é ótima. Entretanto, $T'$ é uma árvore ótima com mais arestas em comum com $T$ do que $T^*$, contradizendo a escolha de $T^*$. Assim, devemos ter $T = T^*$.
\end{proof}
\end{document}
