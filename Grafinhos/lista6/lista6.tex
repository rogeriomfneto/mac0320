\documentclass[11pt,a4paper,notitlepage]{exam}
\usepackage[utf8]{inputenc}
\usepackage{graphicx, wrapfig}

\usepackage{amsmath}
\usepackage{amsthm}
\usepackage{amssymb}
\usepackage{mathtools}
\usepackage[shortlabels]{enumitem}

\renewcommand*{\proofname}{Prova}
% bold math
\usepackage{amsbsy}

% draw pictures (and graphs)
\usepackage{tikz}

% \usepackage[usenames,dvipsnames,svgnames,table]{xcolor}

% code in latex
\definecolor{dkgreen}{rgb}{0,0.6,0}
\definecolor{gray}{rgb}{0.5,0.5,0.5}
\definecolor{mauve}{rgb}{0.58,0,0.82}
\definecolor{newink}{rgb}{0,0.1,0.25}
\usepackage{caption}
\usepackage{listings}
\lstset{frame=tb,
  language=Python,
  aboveskip=3mm,
  belowskip=3mm,
  showstringspaces=false,
  columns=flexible,
  basicstyle={\small\ttfamily},
  numbers=none,
  numberstyle=\tiny\color{gray},
  keywordstyle=\color{blue},
  commentstyle=\color{dkgreen},
  stringstyle=\color{mauve},
  breaklines=true,
  breakatwhitespace=true,
  tabsize=3
}


\usepackage{multirow}

% definition equal
\newcommand\eqdef{\mathrel{\overset{\makebox[0pt]{\mbox{\normalfont\tiny\sffamily def}}}{=}}}

% independence equal
\newcommand\eqindep{\mathrel{\overset{\makebox[0pt]{\mbox{\normalfont\tiny\sffamily indep}}}{=}}}


% independent and identically distributed equal
\newcommand\eqiid{\mathrel{\overset{\makebox[0pt]{\mbox{\normalfont\tiny\sffamily i.i.d.}}}{=}}}

% * to cdot
% \mathcode`\*="8000
% {\catcode`\*\active\gdef*{\cdot}}

% pseudo-code
\usepackage[portuguese, linesnumbered]{algorithm2e}
\newcommand\Recebe{\leftarrow}
\newcommand\Comment{\vartriangleright}
\SetKw{Devolva}{devolva}
% Example:
% \paragraph{}
% \SetAlgoNoLine
% \textsc{Título-Do-Algoritmo}($A, n$)\\
% \begin{algorithm}[H]
%   \Devolva $A$
% \end{algorithm}
%

% pair ceil
\DeclarePairedDelimiter{\ceil}{\lceil}{\rceil}

% pair ceil
\DeclarePairedDelimiter{\floor}{\lfloor}{\rfloor}

% images
\usepackage{graphicx}
\graphicspath{ {./} }
% use: \includegraphics[scale=1]{image}


\setlength{\parindent}{3em}
\setlength{\parskip}{0.5em}

\begin{document}
% \SetAlgoNoLine
\begin{center}
    %NOME E NUSP
    Nome: Rogério Marcos Fernandes Neto\hphantom{xxx} NUSP: 10284632\\
    %CURSO
    Curso: Bacharelado em Ciência da Computação\\
    %MATÉRIA
    MAC0320 - Introdução à Teoria dos Grafos
    \paragraph{}
    \textbf{LISTA 5}
\end{center}
\paragraph{E21.} Prove que uma árvore tem no máximo um emparelhamento perfeito.
\paragraph{Solução:}
\begin{proof}
    Seja $T$ uma árvore. Se $T$ não possui um  emparelhamento perfeito então não há o que provar,
    portanto, suponha que $T$ possui um emparelhamento perfeito $E$.\\
    Suponha, por absurdo, que existe um
    outro emparelhamento $E'$, distinto de $E$. Seja $$H = G[E\triangle E']$$
    Temos $g_H(v) \geq 2$ para
    todo $v\in H$ e, como $T$ é árvore e ambos $E$ e $E'$ são perfeitos, então todas as componentes de
    $H$ devem ser caminhos de comprimeto par. Afirmo que todas as componentes são caminhos de
    comprimento zero. De fato suponha que existe alguma componente $P = (v_1, \dots, v_n)$ que é um
    caminho de comprimento maior que zero, e sem perda de generalidade suponha que $v_1v_2 \in E$. Como
    $P$ tem comprimento par então $v_{n-1}v_n \in E'$ e $v_n$ deve ser um vértice solitário em $E$, o
    que contradiz a hipótese de $E$ ser perfeito. Portanto $H$ não possui arestas. Mas se $H$ não
    possui arestas então
    \begin{align*}
                 & E\triangle E' = \emptyset \\
        \implies & E = E'
    \end{align*}
    contradizendo a hipótese sobre a
    distinção entre $E$ e $E'$, o que completa a prova.
\end{proof}
\paragraph{E22.} Sejam $E$ e $F$ emparelhamentos disjuntos num grafo $G$,
com $|E| > |F|$. Prove que existem emparelhamentos $E'$ e $F'$ tais que
$|E'| = |E| - 1$, $|F'| = |F| - 1$ e  $E'\cup F' = E\cup F$.\\
\paragraph*{Solução:}
\begin{proof}
 Sejam $E$ e $F$ emparelhamentos disjuntos num grafo $G$,
    com $|E| > |F|$. Seja $H := G[E\triangle F]$. Como $g_H(v) \leq 2$
    para todo $v \in H$ então todas as componentes em $H$ são caminhos
    ou circuitos de tamanho par. Como $|E| > |F|$ então existe pelo
    menos uma componente em $H$ que é um caminho $P$ de comprimento
    impar, com mais arestas de $E$ que $F$. Sejam
    \begin{align*}
        E^{*} = E \cap A(P)\\
        F^{*} = F \cap A(P)
    \end{align*}
    Então temos que
    \begin{align*}
        E' = E - E^{*} + F^{*}\\
        F' = F - F^{*} + E^{*}
    \end{align*}
    Como trocamos o emparelhamento de uma componente por outro, então ambos $E'$
    e $F'$ são emparelhamentos. Além disso, como o $P$ tinha tamanho
    ímpar, então temos que $|E'| = |E| - 1$ e $|F'| = |F| + 1$. Por fim,
    como em $E'\cup F'$ foram utilizadas as mesmas arestas que em $E
    \cup F$, então temos que $E \cup F = E' \cup F'$.
\end{proof}
\paragraph*{E23. } Seja $G$ um grafo $(X, Y)-$bipartido com $|X| =
|Y| = n \geq 1$. Prove que se $|A(G)| > n(n-1)$ então $G$ contém
um emparelhamento perfeito. (Sugestão: aplicar Teorema de Hall).
\paragraph*{Solução: }
\begin{proof}
     Seja $G$ um grafo $(X, Y)-$bipartido com $|X| =
    |Y| = n \geq 1$ e $|A(G)| > n(n-1)$. Suponha, por absurdo, que $G$ não tem um emparelhamento
perfeito. Como temos $|X| = |Y|$, isso implica que não existe
    emparelhamento que cobre $X$. Portanto, pelo Teorema de Hall, existe
    algum conjunto $S \subseteq X$ tal que $|Adj(S)| < |S|$. Dessa forma,
    existem no máximo $$|S|*|Adj(S)| \leq |S|*(|S|-1)$$ arestas que ligam $S$ à
    $Y$. Por outro lado, existem no máximo $$(|X|-|S|)|Y| =
    (n-|S|)n$$ arestas que ligam $X-S$ à $Y$. Portanto, ao todo, o
    número de arestas que liga $X$ à $Y$, ou seja, $A(G)$, é menor igual a
    \begin{align*}
        |S|*(|S|-1) + (n-|S|)n &= |S|^2 - |S| + n^2 -|S|n\\
        &=n^2 + |S|(|S| -1 -n)\\
        &\leq n^2 + n(n -1 -n) && \text{$|S|$ é no máximo $|X|=n$}\\
        &= n^2 -n = n(n-1)
    \end{align*}
    o que contradiz a nossa hipótese. Portanto, a afirmação vale.
\end{proof}
    \paragraph*{E24. }Prove que se $G$ é um grafo
    $(X,Y)$-bipartido com pelo menos uma aresta e $g(x)\geq g(y)$
    para todo $x\in X$ e $y\in Y$, então $G$ tem um
    emparelhamento que cobre $X$.
    \paragraph*{Solução:}
    \begin{proof}
        Seja $G$ um grafo $(X, Y)$-bipartido com $A(G)\geq 1$ e $g(x)\geq g(y)$
    para todo $x\in X$ e $y\in Y$. Seja $S\subseteq X$ e defina
        \begin{align*}
            A_1 &= \{a \in A(G): a \text{ incide em } S\}\\
            A_2 &= \{a \in A(G): a \text{ incide em } Adj(S)\}
        \end{align*}
    Claramente, $A_1 \subseteq A_2$ e, portanto, $|A_1| \leq |A_2|$.
    Seja $k = \delta(S)$. Como todo vértice em $S$ possui grau maior
     ou igual a $k$, então temos que $k|S| \leq |A_1|$. Por
     outro lado, como $g(x)\geq g(y)$
    para todo $x\in X$ e $y\in Y$, sabemos que $k =
    \Delta(Adj(S))$ e, portanto, $|A_2|\leq k|Adj(S)|$. Mas então
    temos que 
    \begin{align*}
        & k|S| \leq |A_1| \leq |A_2| \leq k|Adj(S)|\\
        \implies & k|S| \leq k|Adj(S)|\\
        \implies & |S| \leq |Adj(S)|
    \end{align*}
    Portanto, pelo Teorema de Hall, existe um emparelhamento que
    cobre $X$.
    \end{proof}
\paragraph*{E25. } Um \textit{retangulo latino} $m\times n$ é uma
matriz com $m$ linhas e $n$ colunas, cujas entradas são símbolos,
sendo que cada símbolo ocorre no máximo uma vez em cada linha e em cada
colunas. Um \textit{quadrado latino} de ordem $n$ ŕ um retangulo
latino $n\times n$ sobre n símbolos.\medskip\\
Prove: Se $m < n$ então todo retangulo latino $m\times n$ sobre $n$
símbolos pode ser estendido a um quadrado latino de ordem $n$.
\paragraph*{Solução: }
\begin{proof}
    Suponha que temos um retangulo latino $m\times n$, com $m < n$.
    Iremos mostrar que é possível expandir esse retangulo para um
    retangulo latino $(m+1)\times n$.\\
    Vamos construir um grafo $G$ $(X,Y)$-bipartido que
    represente esse retangulo. Sejam:
    \begin{align*}
        X &:= \{1, \dots, n\} && \text{O conjunto dos $n$
        símbolos}\\
        Y &:= \{1, \dots, n\} && \text{O conjunto das $n$ conlunas do
        retangulo}\\ 
    \end{align*}
    e seja:
    $$
    A(G) = \{xy, x \in X, y \in Y: \text{símbolo $x$ ainda não
        ocupou a coluna $y$}\}
    $$
    Como cada símbolo aparece em uma única coluna a cada linha, então
    temos que $g(x) = n-m > 0$, para todo $x\in X$. Pelo mesmo motivo,
    sabemos também que $g(y) = n-m > 0$, para todo $y \in Y$. Portanto,
    temos que $g(y) \geq g(x)$
    para todo $y\in Y$ e $x\in X$, e pela questão \textbf{E24} sabemos
    que existe um emparelhamento que cobre $Y$, isso é, existe uma nova
    reorganização dos $n$ símbolos de modo que não hajam
    conflitos nem nas colunas nem nas linhas do retangulo.
    Portanto, é possível incluir uma nova linha na tabela.
\end{proof}
\end{document}
