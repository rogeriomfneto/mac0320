\documentclass[11pt,a4paper,notitlepage]{exam}
\usepackage[utf8]{inputenc}
\usepackage{graphicx, wrapfig}

\usepackage{amsmath}
\usepackage{amsthm}
\usepackage{amssymb}
\usepackage{mathtools}
\usepackage[shortlabels]{enumitem}

\renewcommand*{\proofname}{Prova}
% bold math
\usepackage{amsbsy}

% draw pictures (and graphs)
\usepackage{tikz}

% \usepackage[usenames,dvipsnames,svgnames,table]{xcolor}

% code in latex
\definecolor{dkgreen}{rgb}{0,0.6,0}
\definecolor{gray}{rgb}{0.5,0.5,0.5}
\definecolor{mauve}{rgb}{0.58,0,0.82}
\definecolor{newink}{rgb}{0,0.1,0.25}
\usepackage{caption}
\usepackage{listings}
\lstset{frame=tb,
  language=Python,
  aboveskip=3mm,
  belowskip=3mm,
  showstringspaces=false,
  columns=flexible,
  basicstyle={\small\ttfamily},
  numbers=none,
  numberstyle=\tiny\color{gray},
  keywordstyle=\color{blue},
  commentstyle=\color{dkgreen},
  stringstyle=\color{mauve},
  breaklines=true,
  breakatwhitespace=true,
  tabsize=3
}


\usepackage{multirow}

% definition equal
\newcommand\eqdef{\mathrel{\overset{\makebox[0pt]{\mbox{\normalfont\tiny\sffamily def}}}{=}}}

% independence equal
\newcommand\eqindep{\mathrel{\overset{\makebox[0pt]{\mbox{\normalfont\tiny\sffamily indep}}}{=}}}


% independent and identically distributed equal
\newcommand\eqiid{\mathrel{\overset{\makebox[0pt]{\mbox{\normalfont\tiny\sffamily i.i.d.}}}{=}}}

% * to cdot
\mathcode`\*="8000
{\catcode`\*\active\gdef*{\cdot}}

% pseudo-code
\usepackage[portuguese, linesnumbered]{algorithm2e}
\newcommand\Recebe{\leftarrow}
\newcommand\Comment{\vartriangleright}
\SetKw{Devolva}{devolva}
% Example:
% \paragraph{}
% \SetAlgoNoLine
% \textsc{Título-Do-Algoritmo}($A, n$)\\
% \begin{algorithm}[H]
%   \Devolva $A$
% \end{algorithm}
%

% pair ceil
\DeclarePairedDelimiter{\ceil}{\lceil}{\rceil}

% pair ceil
\DeclarePairedDelimiter{\floor}{\lfloor}{\rfloor}

% images
\usepackage{graphicx}
\graphicspath{ {./} }
% use: \includegraphics[scale=1]{image}


\setlength{\parindent}{3em}
\setlength{\parskip}{0.5em}

\begin{document}
% \SetAlgoNoLine
\begin{center}
%NOME E NUSP
Nome: Rogério Marcos Fernandes Neto\hphantom{xxx} NUSP: 10284632\\
%CURSO
Curso: Bacharelado em Ciência da Computação\\
%MATÉRIA
MAC0320 - Introdução à Teoria dos Grafos
\paragraph{}
\textbf{LISTA 2}
\end{center}

\paragraph*{E4.}
\begin{enumerate}[(a)]
  \item Prove que um grafo simples de ordem n com mais do que $n^2/4$ arestas não é bipartido.
  \item Encontre todos (diga como são estruturalmente) os grafos bipartidos simples de ordem $n$ com $\floor{n^2/4}$ arestas. Justifique.
\end{enumerate}
\paragraph*{Solução:}
\begin{enumerate}[(a)]
  \item 
  \begin{proof}
    Seja $G$ um grafo de ordem $n$ com mais que $n^2/4$ arestas. Pelo \textbf{teorema de Mantel} $G$ possui um triângulo. Como um triângulo é circuito impar, então pela \textbf{proposição 1.6}, $G$ não é bipartido.
  \end{proof}
  \item Grafos bipartidos simples de ordem $n$ com $\floor{n^2/4}$ são grafos bipartidos completos onde $|X| = \floor{n/2}$ e $|Y| = \ceil{n/2}$ ou $|Y| = \floor{n/2}$ e $|X| = \ceil{n/2}$.
  \begin{proof}
    Iremos mostrar que $G$ é completo. Seja $G$ um grafo de ordem $n$ com uma bipartição $(X, Y)$ e $\floor{n^2/4}$ arestas. Suponha que o grafo $G$ não é completo, isso é, que existem vértices $u \in X$ e $v \in Y$ tais que $uv \notin A(G)$. Como $u$ e $v$ são de partes diferentes, poderiamos acrescentar a aresta $uv$ ao grafo e $G$ continuaria sendo bipartido. Entretanto, agora, $G$ seria um grafo bipartido com mais de $\floor{n^2/4}$ arestas, o que, pelo item \textbf{(a)} é um absurdo. Portanto, $G$ é bipartido completo.\\
    Iremos mostrar que $|X| = \floor{n/2}$ e $|Y| = \ceil{n/2}$ ou vice versa. Como $G$ é completo, o número de arestas no grafo é $|X|\cdot|Y| = \floor{n^2/4}$. Mas $|Y| = n - |X|$, portanto, 
    $$
    |X|\cdot(n - |X|) = \floor{n^2/4} \Longleftrightarrow |X|^2 - n|X| + \floor{n^2/4} = 0
    $$
    Se $n$ é par, então,
    \begin{align*}
      |X|^2 - n|X| + \floor{n^2/4} &= |X|^2 - n|X| + n^2/4\\ 
      &= (|X|-n/2)^2 = 0\\
      &\implies |X| = n/2 = \floor{n/2}\\
      &\hphantom{xxxi} \text{e } |Y| = n/2 = \ceil{n/2}
    \end{align*}
    Se $n$ é impar, então,
    \begin{align*}
      |X|^2 - n|X| + \floor{n^2/4} &= |X|^2 - n|X| + (n^2-1)/4\\ 
      &= (|X|-(n+1)/2)(|X|-(n-1)/2) = 0\\
      &\implies |X| = (n-1)/2 = \floor{n/2}\\
      &\hphantom{xxxi} \text{e } |Y| = (n+1)/2 = \ceil{n/2}\\
      &\hphantom{xxxi} \text{ou}\\
      &\hphantom{xxxxe} |X| = (n+1)/2 = \ceil{n/2}\\
      &\hphantom{xxxi} \text{e } |Y| = (n-1)/2 = \floor{n/2}\\
    \end{align*} 

  \end{proof}
\end{enumerate}

\paragraph{E5.} Um grafo é \textit{auto-complementar} se é simples e é isomorfo ao seu complemento. Mostre que, se $G$ é um  grafo auto-complementar de ordem $n$, então $n \equiv 0 \text{ (mod } 4)$ ou $n \equiv 1 \text{ (mod } 4)$.
\paragraph{Solução:}
  \begin{proof} 
  Seja $G$ um grafo auto-complementar de ordem $n$. Sabemos que $g_G(v) + g_{\bar{G}}(v) = n-1$, portanto $$\sum_{v \in V(G)}g_G(v) + g_{\bar{G}}(v) = n(n-1)$$ Por outro lado, pela \textbf{porposição 1.1} sabemos que
  $\sum_{v \in V(G)}g_(v) + g_{\bar{G}}(v) = 2|A(G)| + 2|A(\bar{G})|$ 
  mas, como $G \cong \bar{G}$, então $|A(G)| = |A(\bar{G})|$ e, portanto,
  $$\sum_{v \in V(G)}g_G(v) + g_{\bar{G}}(v) = 4|A(G)|$$
  Assim, sabemos que 
  $$
  n(n-1) = 4|A(G)| \Longleftrightarrow 4|n \text{ ou } 4|n-1
  $$ 
  ou seja,
  $$
  n \equiv 0 \text{ (mod } 4) \text{ ou } n \equiv 1 \text{ (mod } 4)
  $$
  \end{proof}
  \newpage
\paragraph{E6.} É possível que um grafo auto-complementar de ordem 100 tenha exatamente um vértice de grau 50? Justifique.
\paragraph{Solução:} Não é possível.\\
\begin{proof}
  Seja $G$ um grafo auto-complementar de ordem $n = 100$ com exatamente um vértice $v$ tal que $g_G(v) = 50$. Como $g_G(v) + g_{\bar{G}}(v) = n-1$ Então $v$ também é o único vértice em $\bar{G}$ tal que $g_{\bar{G}}(v) = 49$. Como $G \cong \bar{G}$, então existe um único vértice $u$ em $G$ tal que $g_G(u) = 49$ e, analogamente, $g_{\bar{G}}(v) = 50$. Pelo fato de $u$ e $v$ serem os únicos com seu grau, devemos ter $\varphi(u) = v$ e $\varphi(v) = u$, onde $\varphi$ é a função que define o isomorfismo. Suponha que $vu \in A(G)$, entao devemos ter $\varphi(v)\varphi(u) = uv = vu \in A(\bar{G})$, o que é um absurdo, pois, pela definição de complemento, $uv \in G \Longleftrightarrow uv \notin \bar{G}$. Por outro lado, suponha que $uv \notin V(G)$, isso implica que $uv \in V(\bar{G})$, o que é um absurdo, pois, segundo a definição de isormofismo, $uv \in G \Longleftrightarrow \varphi(v)\varphi(u) = uv = vu \in \bar{G}$.
\end{proof}

\end{document}
