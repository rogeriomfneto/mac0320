
\documentclass[11pt,a4paper,notitlepage]{exam}
\usepackage[utf8]{inputenc}
\usepackage{graphicx, wrapfig}

\usepackage{amsmath}
\usepackage{amsthm}
\usepackage{amssymb}
\usepackage{mathtools}
\usepackage[shortlabels]{enumitem}

\renewcommand*{\proofname}{Prova}
% bold math
\usepackage{amsbsy}

% draw pictures (and graphs)
\usepackage{tikz}

% \usepackage[usenames,dvipsnames,svgnames,table]{xcolor}

% code in latex
\definecolor{dkgreen}{rgb}{0,0.6,0}
\definecolor{gray}{rgb}{0.5,0.5,0.5}
\definecolor{mauve}{rgb}{0.58,0,0.82}
\definecolor{newink}{rgb}{0,0.1,0.25}
\usepackage{caption}
\usepackage{listings}
\lstset{frame=tb,
  language=Python,
  aboveskip=3mm,
  belowskip=3mm,
  showstringspaces=false,
  columns=flexible,
  basicstyle={\small\ttfamily},
  numbers=none,
  numberstyle=\tiny\color{gray},
  keywordstyle=\color{blue},
  commentstyle=\color{dkgreen},
  stringstyle=\color{mauve},
  breaklines=true,
  breakatwhitespace=true,
  tabsize=3
}


\usepackage{multirow}

% definition equal
\newcommand\eqdef{\mathrel{\overset{\makebox[0pt]{\mbox{\normalfont\tiny\sffamily def}}}{=}}}

% independence equal
\newcommand\eqindep{\mathrel{\overset{\makebox[0pt]{\mbox{\normalfont\tiny\sffamily indep}}}{=}}}


% independent and identically distributed equal
\newcommand\eqiid{\mathrel{\overset{\makebox[0pt]{\mbox{\normalfont\tiny\sffamily i.i.d.}}}{=}}}

% * to cdot
% \mathcode`\*="8000
% {\catcode`\*\active\gdef*{\cdot}}
\usepackage[table,xcdraw]{xcolor}
% pseudo-code
\usepackage[portuguese, linesnumbered]{algorithm2e}
\newcommand\Recebe{\leftarrow}
\newcommand\Comment{\vartriangleright}
\SetKw{Devolva}{devolva}
% Example:
% \paragraph{}
% \SetAlgoNoLine
% \textsc{Título-Do-Algoritmo}($A, n$)\\
% \begin{algorithm}[H]
%   \Devolva $A$
% \end{algorithm}
%

% pair ceil
\DeclarePairedDelimiter{\ceil}{\lceil}{\rceil}

% pair ceil
\DeclarePairedDelimiter{\floor}{\lfloor}{\rfloor}

% images
\usepackage{graphicx}
\graphicspath{ {./} }
% use: \includegraphics[scale=1]{image}


\setlength{\parindent}{3em}
\setlength{\parskip}{0.5em}

\usetikzlibrary{graphs,graphs.standard}

\newcount\nodecount
\tikzgraphsset{
  declare={subgraph N}%
  {
    [/utils/exec={\global\nodecount=0}]
    \foreach \nodetext in \tikzgraphV
    {  [/utils/exec={\global\advance\nodecount by1}, 
      parse/.expand once={\the\nodecount/\nodetext}] }
  },
  declare={subgraph C}%
  {
    [cycle, /utils/exec={\global\nodecount=0}]
    \foreach \nodetext in \tikzgraphV
    {  [/utils/exec={\global\advance\nodecount by1}, 
      parse/.expand once={\the\nodecount/\nodetext}] }
  }
}
\linespread{1.2}

\begin{document}
% \SetAlgoNoLine
\begin{center}
    %NOME E NUSP
    Nome: Rogério Marcos Fernandes Neto\hphantom{xxx} NUSP: 10284632\\
    %CURSO
    Curso: Bacharelado em Ciência da Computação\\
    %MATÉRIA
    MAC0320 - Introdução à Teoria dos Grafos
    \paragraph{}
    \textbf{LISTA 9}
\end{center}
    \paragraph{E34.} Mostre que para todo par ($k,l$) tais que $1\leq k
    \leq l$, existe um grafo $G$ tal que $\kappa{G} = k$ e
    $\kappa'(G) = l$.
    \paragraph{Solução:}
    \begin{proof}
        Seja $G_1$ um grafo completo com $2l$ arestas. Seja $G_2$ o grafo
        composto por $G_1$ mais um conjunto $S$ de $k$ vértices, onde cada um dos $k$
        vértices possui um aresta que se conecta a cada um dos vértices
        de $G_1$. Definimos $G$ como sendo o grafo composto por $G_2$
        mais um vértice $v$ que possui $l$ arestas que o conecta
        a vértices de
        $S$.\par
        Temos que $\kappa(G) = k$. De fato, por construção, a única forma de
        tornarmos o grafo desconexo pela remoção de vértices é removendo completamente o conjunto
        $S$. Como $|S| = k$, segue que $\kappa(G)=k$.
        Temos que $\kappa'(G) = l$. De fato, por construção, a única forma de tornar
        esse grafo desconexo pela  remoção de arestas é removendo
        as $l$ arestas que conectam o vértice $v$ à $S$.
        Portanto, $\kappa'(G)=l$.
    \end{proof}  

    \paragraph{E35.} Seja $G$ um grafo $k$-conexo e seja $G'$ o grafo
    obtido de $G$ acrescentando-se um novo vértice e arestas ligando
    esse vértice a todos os vértices de $G$. Prove que $G'$ é
    $(k+1)$-conexo.

    \paragraph{Solução:}
    \begin{proof}
        Seja $G$ um grafo $k$-conexo e seja $G'$ o grafo obtido de $G$
        acrescentando-se um novo vértice $v$ e arestas ligando esse vértice
        a todos os vértices de $G$. \newline
        Suponha, por absurdo, que exista um conjunto separador $S\in V(G')$
        com $k$ vértices e seja $H := G'-S$. Sabemos que $S$ deve conter
        $v$, caso contrário, existe um caminho entre quaisquer dois
        vértices $x$ e $y$ de $H$ dado por $(x,v,y)$, e portanto $S$ não
        seria separador. Sabemos que $G'-v = G$ é conexo. Portanto $S-v$
        deve ser um conjunto separador para $G'-v=G$, mas $|S-v| = k-1$, o
        que contraria nossa hipótese sobre $G$ ser $k$-conexo. Portanto,
        $G'$ é $(k+1)$-conexo.
    \end{proof}

    \paragraph{E36.} Prove que se $G$ é um grafo bipartido $k$-regular
    conexo, então $G$ é $2$-conexo. \smallskip \newline
    [Sugestão: Suponha que $G$ tenha um vértice-de-corte $x$. Então $G =
    G_1 \cup G_2$, e $V(G_1)\cap V(G_2)=\{x\}$. Analise a quantidade de
    arestas em $G_1$ (lembrando que $G$ é $k$-regular) e deduza - por
    essa análise - alguma informação sobre $k$ relativamente a
    $g_{G_1}(x)$, d emodo a obter uma contradição.]
    \paragraph{Solução:}
    \begin{proof}
        Seja $G$ um grafo com uma bipartição $(X,Y)$, $k$-regular e  conexo.\newline
        Suponha, por absurdo, que $G$ possui um vértice de corte $x$.
        Sem perda de generalidade, suponha que $x \in X$.
        Então existem dois conjuntos $G_1$ e $G_2$ tais que $G_1 \cup
        G_2 = G$, e $V(G_1)\cap V(G_2)=\{x\}$. $G_1$ também deve ser
        bipartido. Seja $(X_1, Y_1)$ a bipartição de $G_1$, com $X_1
        \subset
        X$ e $Y_1 \subset Y$.\newline
        Note que devemos ter 
        \begin{equation}
            g_{G_1}(x) < k
        \end{equation}
        pois, caso contrário, $x$
        não seria um vértice de corte. Além disso, o número de arestas
        que partem de $X_1$ para $Y_1$ deve ser igual ao número de
        arestas que parte de $Y_1$ para $X_1$. Como $x\in X_1$, então
        temos que
        \begin{align*}
            (|X_1|-1)k + g_{G_1}(x) &= |Y_1|k\\
            g_{G_1}(x) &= (|Y_1| -|X_1|+1)k\\
        \end{align*}
        Note que devemos ter $|Y_1| - |X_1|+1 \geq 1$, pois $x$ é vértice de
        corte. Mas então temos que
        \begin{equation}
            g_{G_1}(x) \geq k
        \end{equation}
        que, por $(1)$, é um absurdo.
    \end{proof}
    \paragraph*{E37} Prove que se $G$ é grafo $k$-conexo, e
    $k\geq2$, então qualquer conjunto de $k$-vértices de $G$
    pertence a um mesmo circuito de $G$. Tal circuito pode conter outros
    vértices adicionais além dos $k$ vértices fixados.)
    \begin{proof}
        Seja $G = (V,A)$ um grafo $k$-conexo, com $k\geq2$. Seja
        $S \subset V$ um conjunto qualquer de $k$ vértices. \par
        Suponha, por absurdo, que não exista um circuito que
        contenha todos os vértices de $S$. Seja $C = (v_1, \dots,
        v_m)$ um circuito que
        possui a maior quantidade de vértices de vértices de $S$.
        Defina $S_1 := S \cap C$ e $S_2 := S - S_1$. Devemos ter $|S_1|
        \geq 2$, pois como $k \geq 2$ então, pelo \textbf{corolário 8.3}
        entre quaisquer dois vértices $u$ e $w$ de $G$ existem pelo
        menos dois caminhos
        independentes $P_1$ e $P_2$ que ligam $u$ a $w$ e portanto
        $P_1\cdot P_2^{-1}$ é um circuito que possui dois vértices de
        $S$.\par
        Seja $x$ um vértice qualquer de $S_2$ e seja $W\subseteq
        C$ um conjunto de vértices de tamanho $l = |S_1|$ tal que cada caminho
        do leque $x-W$ - como definido e provado existência em aula -
        contenha somente um vértice de $C$. Tal leque divide
        o circuito $C$ em $l+1$ seções da forma $(v_i,\dots,v_j)$
        onde $v_i$ e $v_j$ pertencem ao leque. Portanto, pelo princípio
        da casa dos pombos existe pelo menos uma seção $P = (v_i, \dots,v_j)$ de
        $C$ tal que $P\cap S_1 = \{v_i, v_j\}$ ou $P\cap S_1 =
        \emptyset$. Seja $P_1$ e $P_2$ os caminhos do leque que contém
        $v_i$ e $v_j$ respectivamente. Então
        $(v_1,\dots,v_i)\cdot P_1 \cdot P_2^{-1} \cdot (v_j, \dots,
        v_m)$ é um circuito e contém mais vértces de $S$ do que $C$, o
        que é um absurdo. Portanto deve existir um ciruito que contém
        todos os vértices de $S$.
    \end{proof}

\paragraph{E38.}Seja $G=(V,A)$ um grafo 2-conexo de ordem $n$, e sejam
$v_1$, $v_2$, vértices de $G$. Sejam $n_1$ e $n_2$ inteiros
positivos tais que $n_1+n_2=n$. Mostrque existe uma partição de $V$ em
$V_1\cup V_2$ com $|V_1| = n_1$ e $|V_2| = n2$, tal que $G[V_i]$
é conexo, e $v_i \in V_i$ para  $i=1,2$.
\begin{proof}
    Seja $G=(V,A)$ um grafo 2-conexo de ordem $n$ e sejam $n_1$ e $n_2$
    inteiros positivos tais que $n_1+n_2=n$.\par
    Tome $T$ uma arvore geradora de $G$ e particione $V$ em partes
    $T_1$ e
    $T_2$ com $V(T_1) = V_1$ e $V(T_2)=V_2$ tais que $v_1 \in V_1$ e $v_2 \in V_2$. Se $|V_1| = n_1$
    então não há o que provar. Portanto, suponha sem perda de
    generalidade que $|V_1| > n_1$. Nesse caso, iremos aplicar um
    procedimento a fim de pegar vértices de $V_2$ e tranferí-los para
    $V_1$.\par
    Enquanto tivermos $|V_1| > n_1$ faça o seguinte:
    \begin{enumerate}
        \item Se existe uma folha $v$ diferente de  de $T_1$ vizinha de algum vértice de $T_2$
            faça:
        \item \hspace*{20px} $T_1 \Recebe T_1 -v \qquad T_2 \Recebe
            T_2 + v$
        \item senão faça:
        \item    \hspace*{20px} Seja $x$ um vértice de $T_1$  mais distante
            possível de $v_1$ que possui algum vizinho de $T_2$.
            Sabemos que tal vértice existe e é distinto de $v_1$ pois, caso contrário, então $v_1$ seria de corte em $G$, o que
            contradiz a hipótese sobre $G$.
        \item \hspace*{20px} Tome $x$ com raiz de $T_1$. Como $x$ não
            é folha, então $x$ deve ter pelo menos duas
            subárvores distintas. Sabemos também que nenhuma dessas
            subárvores possui um vértice $w$ que possui
            vizinhos em $T_2$, pois, caso tivesse, contradiria a
            hipótese sobre $x$ ser mais distante de $v_1$.
        \item \hspace*{20px} Para cada subárvore $T_{1_i}$ de $x$
            faça:
        \item \hspace*{40px} Se $T_{1_i}$ não contém $v_1$ faça:
        \item \hspace*{60px} Tome o vértice $u$ filho de $x$.
            Como $u$ não possui vizinhos em $T_2$ (linha 5), então
            $u$ possui vizinho em outra sub
            $T_{1_j}$.
        \item \hspace*{60px} Tome $k$ um vértice
            vizinho de $u$ em $T_{1_j}$.
        \item \hspace*{60px} $x \Recebe x -T_{1_i} \qquad k \Recebe k +
            T_{1_1}\qquad \Comment \text{Transfere a subárvore de $x$
            para $k$}$
        \item \hspace*{40px} $T_1 \Recebe T_1 -x \qquad T_2 \Recebe
            T_2 + x$
    \end{enumerate}

    Ao fim da última iteração do laço da linha 9, $x$ contém apenas
    uma subárvore: a que contém $v_1$. Portanto, agora é possível
    transferi-lo para a árvore $T_2$ sem perder a conexidade.
    Repetindo esse procedimento até atingirmos $|V_1 = n_1|$
    conseguimos a árvore específicada no enunciado.
\end{proof}

\end{document}
