\documentclass[11pt,a4paper,notitlepage]{exam}
\usepackage[utf8]{inputenc}
\usepackage{graphicx, wrapfig}

\usepackage{amsmath}
\usepackage{amsthm}
\usepackage{amssymb}
\usepackage{mathtools}
\usepackage[shortlabels]{enumitem}

\renewcommand*{\proofname}{Prova}
% bold math
\usepackage{amsbsy}

% draw pictures (and graphs)
\usepackage{tikz}

% \usepackage[usenames,dvipsnames,svgnames,table]{xcolor}

% code in latex
\definecolor{dkgreen}{rgb}{0,0.6,0}
\definecolor{gray}{rgb}{0.5,0.5,0.5}
\definecolor{mauve}{rgb}{0.58,0,0.82}
\definecolor{newink}{rgb}{0,0.1,0.25}
\usepackage{caption}
\usepackage{listings}
\lstset{frame=tb,
  language=Python,
  aboveskip=3mm,
  belowskip=3mm,
  showstringspaces=false,
  columns=flexible,
  basicstyle={\small\ttfamily},
  numbers=none,
  numberstyle=\tiny\color{gray},
  keywordstyle=\color{blue},
  commentstyle=\color{dkgreen},
  stringstyle=\color{mauve},
  breaklines=true,
  breakatwhitespace=true,
  tabsize=3
}


\usepackage{multirow}

% definition equal
\newcommand\eqdef{\mathrel{\overset{\makebox[0pt]{\mbox{\normalfont\tiny\sffamily def}}}{=}}}

% independence equal
\newcommand\eqindep{\mathrel{\overset{\makebox[0pt]{\mbox{\normalfont\tiny\sffamily indep}}}{=}}}


% independent and identically distributed equal
\newcommand\eqiid{\mathrel{\overset{\makebox[0pt]{\mbox{\normalfont\tiny\sffamily i.i.d.}}}{=}}}

% * to cdot
% \mathcode`\*="8000
% {\catcode`\*\active\gdef*{\cdot}}
\usepackage[table,xcdraw]{xcolor}
% pseudo-code
\usepackage[portuguese, linesnumbered]{algorithm2e}
\newcommand\Recebe{\leftarrow}
\newcommand\Comment{\vartriangleright}
\SetKw{Devolva}{devolva}
% Example:
% \paragraph{}
% \SetAlgoNoLine
% \textsc{Título-Do-Algoritmo}($A, n$)\\
% \begin{algorithm}[H]
%   \Devolva $A$
% \end{algorithm}
%

% pair ceil
\DeclarePairedDelimiter{\ceil}{\lceil}{\rceil}

% pair ceil
\DeclarePairedDelimiter{\floor}{\lfloor}{\rfloor}

% images
\usepackage{graphicx}
\graphicspath{ {./} }
% use: \includegraphics[scale=1]{image}


\setlength{\parindent}{3em}
\setlength{\parskip}{0.5em}

\usetikzlibrary{graphs,graphs.standard}

\newcount\nodecount
\tikzgraphsset{
  declare={subgraph N}%
  {
    [/utils/exec={\global\nodecount=0}]
    \foreach \nodetext in \tikzgraphV
    {  [/utils/exec={\global\advance\nodecount by1}, 
      parse/.expand once={\the\nodecount/\nodetext}] }
  },
  declare={subgraph C}%
  {
    [cycle, /utils/exec={\global\nodecount=0}]
    \foreach \nodetext in \tikzgraphV
    {  [/utils/exec={\global\advance\nodecount by1}, 
      parse/.expand once={\the\nodecount/\nodetext}] }
  }
}
\linespread{1.2}

\begin{document}
% \SetAlgoNoLine
\begin{center}
    %NOME E NUSP
    Nome: Rogério Marcos Fernandes Neto\hphantom{xxx} NUSP: 10284632\\
    %CURSO
    Curso: Bacharelado em Ciência da Computação\\
    %MATÉRIA
    MAC0320 - Introdução à Teoria dos Grafos
    \paragraph{}
    \textbf{LISTA 8}
\end{center}
\paragraph*{E30.}Seja $G$ um grafo simples com $n$ vértices, e seja
$\alpha$ a cardinalidade de um conjunto independente máximo de $G$. Prove
que\bigskip\newline
(a)$\frac{n}{\alpha} \leq \chi(G) \leq n - \alpha + 1$.\newline
(b) Caracterize (diga como são) os grafos $G$ de ordem $n$ tais que
$\chi(G)= n -\alpha + 1$. 
\paragraph*{Solução: }
\textbf{a)}
\begin{proof}
Seja $G$ um grafo de cardinalidade $n$ e seja $\alpha$ a cardinalidade
de um conjunto independente máximo de $G$. Sabemos que existe uma
partição $\{ X_1, X_2, \dots, X_{\chi(G)} \}$ dos vértices de $G$ em
conjuntos independentes. Assim, 
\begin{align*}
    n &= \sum_{i=1}^{\chi(G)}|X_i|\\
    n &\leq \sum_{i=1}^{\chi(G)}\alpha && \text{pois $|X_i| \leq
    \alpha$}\\
    n &\leq \chi(G)\cdot \alpha\\
    \implies \dfrac{n}{\alpha} &\leq \chi(G)
\end{align*}

Por outro lado, sabemos que com exceção de alguma parte $X_j$ com
tamanho $\alpha$, todas as outras
tem tamanho pelo menos $1$, isso é, $|X_i| \geq 1$, $i \neq j$. Portanto
\begin{align*}
    n &= \sum_{i=1}^{\chi(G)}|X_i|\\
    n &= \sum_{i=1,i\neq j}^{\chi(G)}|X_i| + |X_j|\\
    n &\geq \sum_{i=1,i\neq j}^{\chi(G)}1  + \alpha && \text{pois $|X_i|
    \geq 1$ para $i\neq j$}\\
    n &\geq (\chi(G) -1) + \alpha\\
    \implies \chi(G) &\leq n - \alpha + 1
\end{align*}
\end{proof}
\textbf{b)} Grafos dessa forma possuem uma \textit{click} com $n-\alpha$
vértices e outros $\alpha$ vértices se conectam com todos os vértices da
click mas não com eles mesmos. Mais formalmente, um grafo $G=(V, A)$ dessa
forma é definido por
\begin{align*}
V &:= \{v_1, \dots, v_n\}\\
A &:= \{ \{v_iv_j\}: i,j < \alpha \text{ ou  } i<\alpha,j \geq \alpha\}
\end{align*}

Note que dessa forma os $n-\alpha$ primeiros vértices formam uma click e
os $\alpha$ ultimos vértices formam um conjunto independente de tamanho
$\alpha$. Portanto, se particionarmos $V$ no menor número de componentes
independentes possível, teremos uma componente de tamanho $\alpha$ e
$\chi(G) -1$ componentes de tamanho $1$. Por construção, não existe
outra opção de componentes de tamanho $\chi(G)$. Portanto temos que
\begin{align*}
    n = \chi(G) -1 + \alpha \implies \chi(G) = n -\alpha + 1
\end{align*}

\paragraph*{E31.} Seja $G$ um grafo de ordem $n$. Prove, por
indução em $n$, que $\chi(G) + \chi(\bar{G}) \leq n + 1$.

\paragraph*{Solução:}
\begin{proof}
    Seja $G$ um grafo de ordem $n$. Por indução em $n$ iremos mostrar
    que $\chi(G) + \chi(\bar{G}) \leq n + 1$.\newline
    \textbf{Base:} Suponha $n = 1$. Nesse caso, claramente tanto $G$
    quando $\bar{G}$ são coloridos com uma única cor e portanto temos
    que $\chi(G) + \chi(\bar{G}) =  1 + 1 \leq n + 1$. \newline
    \textbf{Passo:} Suponha agora que $\geq 2$ e que a afirmação
    vale para grafos com até $n-1$ vértices.
    Seja $v$ um vértice qualquer de $G$ e seja $H := G - v$. Como
    $|V(H)| = n-1$ então, por hipótese, sabemos que $\chi(H) +
    \chi(\bar{H}) \leq (n-1) + 1 = n$. Sejam $\{X_1, \dots,X_k\}$
    e$\{X'_1, \dots, X'_{k'} \}$ colorações de $H$ e $\bar{H}$
    respectivamente. 
    \begin{enumerate}[a)]
        \item  Se $\chi(H) + \chi(\bar{H}) \leq n-1$ podemos atribuir
            a $v$ uma cor distinta das $k$ cores utilizadas em $H$, e
            outra cor distinta das $k'$ cores utilizadas em
            $\bar{H}$. Assim obtemos as colorações $\{\{v\}, X_1, \dots,X_k\}$ e
            $\{\{v\}X'_1, \dots, X'_{k'} \}$ para $G$ e $G'$
            respectivamente. Portanto
            \begin{align*}
                \chi(G) + \chi(\bar{G}) = (\chi(H) + 1) +
                (\chi(\bar{H}) + 1) \leq n-1 + 2 = n+1
            \end{align*}

        \item Se $\chi(H) + \chi(\bar{H}) = n$, então afirmo que se
            for atribuída uma nova cor a $v$ para colorir $G$ então
            não será atribuída uma nova cor para colorir $v$ em
            $\bar{G}$ e vice-versa. De fato, suponha que sejam
            atribuídas novas cores para $v$ em ambas as
            colorações. Então temos que $g_H(v)\geq k$ e
            $g_{\bar{H}}(v)\geq k'$. Mas sabemos também que $g_H(v)
            + g_{\bar{H}}(v) = n-1$. Portanto, temos que $$\chi(H) +
            \chi(\bar{H}) = k + k' \leq g_H(v)
            + g_{\bar{H}}(v) = n-1$$
            o que contradiz nossa hipótese. Portanto sabemos que
            $\chi(G) +
            \chi(\bar{G}) \leq \chi(H) +
            \chi(\bar{H}) + 1 = n + 1$.
    \end{enumerate}
    Portanto, pelo princípio da indução, a afirmação vale.
\end{proof}

\paragraph*{E32.} Seja $G$ um grafo que tem uma coloração própria
(de seus vértices) na qual toda cor é usada pelo menos 2 vezes. Mostre que
que $G$ que tem uma coloração (de seus vérties) com $\chi(G)$ cores
que tem essa mesma propriedade.

\paragraph{Solução:}
\begin{proof}
    Seja $G$ um grafo que possui uma coloração própria  $C =
    \{X_1,\dots, X_k \}$ onde toda cor é usada pelo menos duas vezes.
    Seja $C' = \{X'_1,\dots, X'_{k'} \}$ uma coloração qualquer de $G$
    com $\chi(G) = k'$ cores. Se todas as cores de $C'$ são utilizadas
    duas vezes então não há o que mostrar. Portanto, suponha que exista
    pelo menos uma cor que possui um único vértice. Iremos repetir o
    seguinte procedimento a fim de adequar a coloração $C'$:
    \begin{enumerate}[a)]
        \item Tome $v$ um vértice que seja o único de sua cor. Caso não
            exista tal vértice, pare.
        \item Seja $X_j$ o conjunto de vértices que tem a mesma cor
            que $v$ em $C$
        \item Pinte com a mesma cor de $v$ em $C'$ todos vértices de $X_j$.   
    \end{enumerate}
    Alguns fatos:
    \begin{enumerate}
        \item Note que o conjunto de vértices de $X_j$ podem ser pintados com a
         mesma cor de $v$ uma vez que $v$ era o único de sua cor e todo
         vértice em $X_j$ não é adjacente.
        \item Vértices que ja foram repintados não serão repintados
            novamente, pois como vimos, todos os vértices de $X_j$ foram
            pintados com a mesma cor e, como $|X_j| \geq 2$, não existem vértices
            solitários pertecentes a $X_j$.
        \item A quantidade de cores não é alterada pois são utilizadas
            cores ja presentes nos vértices.
        \item $G$ é finito.
    \end{enumerate}
    Devido a esses fatos o algoritmo para e o algoritmo está
    correto.

    \paragraph{E33.} Sejam $I_1, I_2, \dots, I_n$ intervalos
    fechados na reta real. Seja $G$ o grafo simples com vértices
    $v_1, v_e, \dots, v_n$ tal que para todo $i$, $j$,
    $$
    v_i \text{ é adjacente a } v_j \text{ se e só se } I_i\cap I_j
    \neq \emptyset
    $$
    Mostre que $\chi(G) = \omega(G)$. (Lembramos que uma
    \textit{clique} é um subgrafo completo, e $\omega(G)$ denora a
    cardinalidade de uma clique máxima em G) \textbf{Sugestão:} indução
    em $n$. Remova um intervalo que tem o menor extremo
    superior.\medskip\newline
    OBS: O grafo G acima definido é chamado de \textit{grafo de
    intervalos}.
\end{proof}

\paragraph{Solução:}

\begin{proof}
    Seja $G$ um grafo de intervalos de ordem $n$.\newline
    \textbf{Base:} Suponha $n=1$. Nesso caso é claro que $\chi(G)
    = \omega(G) = 1$ e portanto a afirmação vale.\newline
    \textbf{Passo:} Suponha $n \geq 2$ e que a afirmação vale para
    grafos de intervalos de ordem até $n-1$. Seja $I_j$ um
    intervalo com menor extremo inferior e seja $H := G-v_j$. Por
    hipótese temos que $\chi(H) = \omega(H)$, e portanto existe uma
    coloração $\{X_1,\dots,X_{k}\}$  dos vértices de $H$.
    Temos duas situações:
    \begin{enumerate}[a)]
        \item $\omega(G) = \omega(H) + 1$. Nessa caso, basta atribuir
            uma nova cor a $v_j$ diferente das $k$ cores
            utilizadas em $H$. Assim, $\{{v_j},X_1,\dots,X_{k} \}$ é
            uma coloração para $G$ com $\chi(G) = k+1 = \omega(H)
            + 1 = \omega(G)$ cores.
        \item $\omega(G) = \omega(H)$. Nesse caso, $v_j$ tem, no
            máximo, $\omega(G)-1$ vizinhos. De fato, como $I_j$ tem
            extremo superior mínimo, então todo intervalo que
            intersecta $I_j$ deve conter seu extremo superior. Caso
            contrário teríamos um intervalo com extremo superior menor
            que de $I_j$. Mas então $v_j$ e seus vizinhos formam
            uma clique, pois todo intervalo $I_i$ que
            intersecta $I_j$ tem como ponto comum o extremo superior
            de $I_j$. Como sabemos que a maior clique tem tamanho
                $\omega(G)$, então $I_j$ tem no máximo $\omega(G) -1$
            vizinhos. Portanto, basta atribuir a $v_j$ uma cor distinta
            daquela de seus vizinhos. Assim, é possível colorir
            $G$ com $\omega(G)$ cores.
    \end{enumerate}
    Portanto, pelo princípio da indução afirmação vale.
\end{proof}

\paragraph{B7.} Seja $G$ um grafo tal que todo par de circuitos ímpares
tem (pelo menos) um vértice em comum. Mostre que que $G$ tem uma
5-coloração.

\paragraph{Solução:}
\begin{proof}
    Seja $G$ um grafo tal que todo par de circuitos ímpares tem pelo
    menos um vértice em comum. Seja $C$ um circuito ímpar qualquer
    em $G$ e seja $H:=G-C$. Como todo circuito ímpar em $G$ tem pelo
    menos um vértice em comum com $C$, então $H$ não tem circuitos ímpares.
    Portanto $H$ é bipartido. Dessa forma, é possível colorir $H$ com
    2 cores, digamos $\{X_1, X_2\}$. Como $C$ é um circuito impar então
    possível colorir $C$ com 3 cores, digamos $\{X_3, X_4, X_5\}$.
    Mas então $\{X_1, \dots, X_5\}$ é uma coloração para $G$. De fato,
    $\{X_1, \dots, X_5\}$ partição de $G$ em conjuntos
    independentes, pois $G$ apenas acrescenta arestas que ligam $C$ à
    $H$ e não arestas que ligam $H$ à $H$ ou $C$ à $C$. Dessa forma, a
    independencia entre os conjuntos é mantida.
\end{proof}
\end{document}

